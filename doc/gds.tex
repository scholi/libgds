\documentclass{article}

\usepackage[utf8]{inputenc}
\usepackage{fullpage}
\usepackage{listings}
\usepackage{txfonts}
\usepackage{fourier-orns}
\usepackage{fancybox}
\usepackage{array}
\usepackage{textcomp}

\newcommand{\cmd}[1]{\texttt{#1}}
\newcommand{\var}[1]{\textit{#1}}
\newcommand{\war}[2][14.5]{\ovalbox{
\begin{tabular}{m{1cm}m{#1cm}}
\Huge\danger & #2
\end{tabular}
}}

\author{Olivier Scholder}
\title{Manual of the GDS.py Library}

\begin{document}
\maketitle
\tableofcontents
\section{Installation}
The installation is easy, just copy the script wherever you want and assign the environement variable PYTHONPATH to the directory path where you put the file.
\paragraph{ToDo}$\frac{}{}$\\
\lstset{language=bash}
\begin{lstlisting}[frame=single]
cd
mkdir scripts
cp PATH_TO_GDS/gds.py scripts/
echo 'export PYTHONPATH="~/scripts"' >> ~/.bashrc
\end{lstlisting}

\section{General options}
All the units are set in nm if not specified.
\begin{description}
\item[layer] On which layer you want your line
\item[width] The line width. For Raith is normaly set to 0. It's the dose which will actually define your width
\item[dose] The dose set the DATATYPE variable which is interpreted by Raith as the dose.
\item[loop] Tell how many time in a row the object should be looped. If set to \var{None} (which is default), uses the internal \var{loops} variable.\\
\war[13.5]{This function works only for FIB (Elphy Multibeam). This tag will be probably discarded by Raith150 II or other system, meaning that only a single loop will be performed.}
\end{description}

\section{Basic Functions}
\subsection{First step}
The very first step is to call the library:
\lstset{language=python}
\begin{lstlisting}[frame=single]
#!/usr/bin/python

import gds

g=gds.GDSII()
\end{lstlisting}
This small piece of code will import you gds library and create a new GDSII object called g. This steps just prepare everything you need to write GDSII files. By calling \cmd{GDSII()} without argument, this will store all of your data in memory until you call the \cmd{save()} function. An alternative to save memory, as you are preparing large data, is to give in argument the file name of the gds you want to write into:
\begin{lstlisting}[frame=single]
#!/usr/bin/python

import gds

g=gds.GDSII("mygds.gds")

# ...
# Add structures ...
# ...
g.close()
\end{lstlisting}
Notice, that you need to close that file with the command \cmd{close()} with this direct writing method.
\par\noindent
\war{The reader should also be aware that this script never check if file already exists an will overwrite already existing files.}

\subsection{Create new file}
\paragraph{Headers}
The GDSII object provides you the \cmd{new()} function which creates automatically the header of your GDS file. As argument new take the name of your library, which has mainly no importance. Normally \cmd{new()} should be called just after the GDSII object creation.
\paragraph{Footers and writing file}
In order to save your structure to a GDS file you should write the footers which are done automatically by calling the \cmd{endLib()} function. Once this step is done you can save your GDS file with the \cmd{write()} function. You have to specifiy the path of your gds as argument to the \cmd{write()} function.
\paragraph{Simple Example}
The following example will write a valid GDSII file containing no structures. It should be the basic of all your scripts.
\begin{lstlisting}[frame=single]
#!/usr/bin/python

import gds

g=gds.GDSII()
g.new('EMPTY')
g.endLib()
g.write('empty.gds')
\end{lstlisting}

\section{Graphical Functions}
Now let's see what we can do graphically with the GDSII files
\subsection{Lines}
\verb?addLine(pts, layer=0, width=0, dose=1, loop=None)?
\begin{description}
\item[pts] the points list. pts=[x1,y1,x2,y1,...,xn,yn]
\end{description}

\subsection{Circles and Arcs}
\verb?addCircle(pos, radius, npts=10, layer=0, width=0, dose=1, A=0, B=360, loop=None)?
\begin{description}
\item[pos] The position of the center of the circle. \var{pos}=[x,y]
\item[radius] The radius of the circle
\item[npts] The circle is actually a regular polygon and npts specify the number of submits. So if \var{npts}=4, you will get a square.
\item[A] The start angle
\item[B] The end angle. So if \var{A}=0 and \var{B}=90, you will get the quarter of a circle
\end{description}

\subsection{Polygons}
\verb?addPoly(pos, layer=0, dose=1, loop=None)?
\begin{description}
\item[pos] list of points. \var{pos}=[x1,y1,x2,y2,...,xn,yn]
\end{description}

\subsection{Rectangles}
\verb?addRect(pos, layer=0, dose=1, CCW=False, loop=None)?
\begin{description}
\item[pos] \var{pos}=[xll,yll,width,height]
\item[CCW] draw ClowckWise or CounterClockWise=?
\end{description}

\subsection{Text}
\verb?addText(pos, txt, height=None, mag=22.22222, layer=0, width=0, angle=0, loop=None)?
\begin{description}
\item[pos] position of the text. \var{pos}=[x,y]. Normaly it's the middle point of the text.
\item[txt] The text as string.
\item[height] Set the text height. This option will compute automatically the corresponding magnification and will override the option mag.
\item[mag] The magnification. A \var{mag} of 22.22222 will give a text of $1\,\hbox{\textmu}\rm{m}$ height. It's maybe better to use the height option.
\item[angle] The rotation angle of the text.
\end{description}

\section{Dose and Loops}
\subsection{Dose}


\section{Examples}
Examples are always much better than words.
\begin{lstlisting}[breaklines=true,frame=single]
#!/usr/bin/python

import gds

g=gds.GDSII()
g.new('DEMO')
g.newStr('Dipant')
space=(1000,1000)
for y in range(5):
	for x in range(5):
		g.addLine((x*space[0],y*space[1],x*space[0]+width,y*space[1]),dose=y*0.5+x*0.1)
for y in range(5):
	g.addText((-space[0],y*space[1]),str(y*0.5),height=500)
for x in range(5):
	g.addText((x*space[0],-space[1]),"+"+str(x*0.1),height=500)
g.endStr()
g.endLib()
g.write('demo.gds')
\end{lstlisting}
\end{document}
